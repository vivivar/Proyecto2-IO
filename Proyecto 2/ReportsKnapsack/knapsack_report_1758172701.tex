\documentclass{article}
\usepackage[utf8]{inputenc}
\usepackage{amsmath}
\usepackage{xcolor}
\usepackage{colortbl}
\usepackage{geometry}
\usepackage{multirow}
\usepackage{graphicx}
\geometry{margin=1in}
\definecolor{verde}{RGB}{0, 128, 0}
\definecolor{rojo}{RGB}{255, 0, 0}
\title{Proyecto 1: Rutas Optimas (Algoritmo de Floyd)}
\author{Emily Sanchez \\ Viviana Vargas \\[1cm] Curso: Investigación de Operaciones \\ II Semestre 2025}
\date{\today}

\begin{document}
\maketitle

\thispagestyle{empty}
\newpage
\setcounter{page}{1}

\section{Problema de la Mochila (Knapsack Problem)}

El \textbf{problema de la mochila} es un clasico de la \textit{optimizacion combinatoria}. Se dispone de una \textbf{mochila} con una \textbf{capacidad maxima} $W$ y un conjunto de $n$ objetos. Cada objeto $i$ tiene un \textbf{peso} $w_i$ y un \textbf{valor} $v_i$. El objetivo es seleccionar los objetos de manera que:
\begin{itemize}
  \item La suma total de los pesos no exceda la capacidad $W$.
  \item Se maximice el valor total de los objetos elegidos.
\end{itemize}

\subsection{Variantes principales}
\begin{description}
  \item[0/1 Knapsack] Cada objeto puede elegirse una sola vez o no elegirse: decision binaria.
  \item[Bounded Knapsack] Cada objeto puede seleccionarse un numero limitado de veces.
  \item[Unbounded Knapsack] Se permite una cantidad ilimitada de cada objeto.
\end{description}

\subsection{Solucion}
\paragraph{0/1 Knapsack} Se resuelve comunmente con \textbf{programacion dinamica}. Sea $dp[i][w]$ el valor maximo al considerar los primeros $i$ objetos y capacidad $w$.
\[
dp[i][w] =
\begin{cases}
dp[i-1][w] & \text{si } w_i > w, \\
\max ( dp[i-1][w], v_i + dp[i-1][w - w_i] ) & \text{si } w_i \le w.
\end{cases}
\]

\paragraph{Unbounded Knapsack} Similar al 0/1 pero permitiendo repeticiones:
\[
dp[w] = \max ( dp[w], v_i + dp[w - w_i] ).
\]

\thispagestyle{empty}
\newpage
\textbf{Tipo de problema:} 0/1 Knapsack\\
\textbf{Capacidad máxima:} 11\\
\textbf{Número de objetos:} 3\\

\clearpage
\section*{Datos del Problema}
\begin{tabular}{|c|c|c|c|}
\hline
Objeto & Costo & Valor & Cantidad \\
\hline
A & 5,00 & 8,00 & 1 \\
B & 3,00 & 5,00 & 1 \\
C & 2,00 & 4,00 & 1 \\
\hline
\end{tabular}

\section*{Tabla de Programación Dinámica}
\begin{center}
\scriptsize
\begin{tabular}{|c|c|c|c|c|}
\hline
Capacidad/Objetos & Ninguno & A & B & C \\ \hline
0 & \cellcolor{rojo}\textcolor{white}{0} & \cellcolor{rojo}\textcolor{white}{0} & \cellcolor{rojo}\textcolor{white}{0} & \cellcolor{rojo}\textcolor{white}{0} \\ \hline
1 & \cellcolor{rojo}\textcolor{white}{0} & \cellcolor{rojo}\textcolor{white}{0} & \cellcolor{rojo}\textcolor{white}{0} & \cellcolor{rojo}\textcolor{white}{0} \\ \hline
2 & \cellcolor{rojo}\textcolor{white}{0} & \cellcolor{rojo}\textcolor{white}{0} & \cellcolor{rojo}\textcolor{white}{0} & \cellcolor{verde}\textcolor{white}{4} \\ \hline
3 & \cellcolor{rojo}\textcolor{white}{0} & \cellcolor{rojo}\textcolor{white}{0} & \cellcolor{verde}\textcolor{white}{5} & \cellcolor{rojo}\textcolor{white}{5} \\ \hline
4 & \cellcolor{rojo}\textcolor{white}{0} & \cellcolor{rojo}\textcolor{white}{0} & \cellcolor{verde}\textcolor{white}{5} & \cellcolor{rojo}\textcolor{white}{5} \\ \hline
5 & \cellcolor{rojo}\textcolor{white}{0} & \cellcolor{verde}\textcolor{white}{8} & \cellcolor{rojo}\textcolor{white}{8} & \cellcolor{verde}\textcolor{white}{9} \\ \hline
6 & \cellcolor{rojo}\textcolor{white}{0} & \cellcolor{verde}\textcolor{white}{8} & \cellcolor{rojo}\textcolor{white}{8} & \cellcolor{verde}\textcolor{white}{9} \\ \hline
7 & \cellcolor{rojo}\textcolor{white}{0} & \cellcolor{verde}\textcolor{white}{8} & \cellcolor{rojo}\textcolor{white}{8} & \cellcolor{verde}\textcolor{white}{12} \\ \hline
8 & \cellcolor{rojo}\textcolor{white}{0} & \cellcolor{verde}\textcolor{white}{8} & \cellcolor{verde}\textcolor{white}{13} & \cellcolor{rojo}\textcolor{white}{13} \\ \hline
9 & \cellcolor{rojo}\textcolor{white}{0} & \cellcolor{verde}\textcolor{white}{8} & \cellcolor{verde}\textcolor{white}{13} & \cellcolor{rojo}\textcolor{white}{13} \\ \hline
10 & \cellcolor{rojo}\textcolor{white}{0} & \cellcolor{verde}\textcolor{white}{8} & \cellcolor{verde}\textcolor{white}{13} & \cellcolor{verde}\textcolor{white}{17} \\ \hline
11 & \cellcolor{rojo}\textcolor{white}{0} & \cellcolor{verde}\textcolor{white}{8} & \cellcolor{verde}\textcolor{white}{13} & \cellcolor{verde}\textcolor{white}{17} \\ \hline
\end{tabular}
\end{center}
\normalsize

\section*{Solución Óptima}
\textbf{Valor máximo obtenido:} 17\\
\textbf{Objetos seleccionados:} C, B, A\\
\textbf{Capacidad utilizada:} 10\\
\end{document}
